\documentclass[twocolumn]{article}
\usepackage{graphicx}

\begin{document}
\title{Roots of tan(x)}
\author{Flemming - 201508887}
\date{}
\maketitle
\begin{abstract}
We will find some roots in tan($x$).
\end{abstract}

\section{explaination}

The invers tangent can be found numerically by determining the roots of tan$^{-1}(a)-x=0$. 
This can in turn be done with the gsl one or multidimensional root-finding routines. 
The most efficient choice is presumably to use the one-dimensional root-finder with derivatives. 
However, we haven't chosen that method since we the program anyway ran so quickly that speed was not an issue.
Instead we have chosen the \texttt{gsl\_multiroot\_fsolver\_hybrids} routine.

\section{illustration}
A graph of the inverse tangent can be seen on figure \ref{plot}.

\begin{figure}[h!]
\input{plot-cairo.tex}
\label{plot}
\end{figure}


\end{document}

